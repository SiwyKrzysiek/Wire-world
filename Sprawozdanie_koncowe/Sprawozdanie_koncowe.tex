\documentclass{mwart}
\usepackage{multicol}
\usepackage{polski} % Pozwala na użycie polskiego. Ustawia między innymi fontenc na T1
\usepackage[utf8]{inputenc} % Informuje o kodowaniu
\usepackage{enumitem}
\usepackage{xcolor}
\usepackage{xcolor}% http://ctan.org/pkg/xcolor
\usepackage{hyperref}
\usepackage{listings}
\usepackage{float} % Ustawianie obrazów
\usepackage[caption = false]{subfig} % Wiele obrazów w jednej figurze
\definecolor{LinkColor}{HTML}{1d5cc1}
\renewcommand{\labelitemi}{\textbullet} % Zmiana symbolu wliczeń

\lstset{
  basicstyle=\ttfamily,
  columns=fullflexible,
  breaklines=true,
  postbreak=\mbox{\textcolor{red}{$\hookrightarrow$}\space},
}

\definecolor{LinkColor}{HTML}{1d5cc1}

\usepackage{tabto}

\usepackage{graphicx} % Pakiet do obrazów
\graphicspath{ {./Obrazy/} } % Folder, z którego będą brane obrazy

% Nie twórz nowych stron
\usepackage{etoolbox}
\makeatletter
% \patchcmd{\chapter}{\if@openright\cleardoublepage\else\clearpage\fi}{}{}{}
\makeatother

\newcommand{\paragraphnl}[1]{\paragraph{#1} \mbox{} \\} % Paragraf z nową linią

\title{PRIR Sprawozdanie końcowe -- Gra w życie}
\author{Krzysztof Dąbrowski 293101}
\date{\today}

\begin{document}
\maketitle{}

\tableofcontents{}

\section{Opis projektu}
W ramach projektu został zmodyfikowany symulator automatów komórkowych \textit{Wireworld} i \textit{Gra w życie} napisany oryginalnie w ramach przedmiotu \texttt{JIMP2}. Przyjęta forma zadania pozwoliła na rozwój umiejętności związanych z utrzymaniem kodu. Rozbudowa projektu o zrównoleglenie obliczeń dobrze wpisuję się w typowe zastosowanie programowania równoległego, które często nie zmienia funkcjonalności programu, lecz pozwala przyśpieszyć zastosowane rozwiązania lub pośrednio umożliwić przetwarzanie większych zbiorów danych.

\section{Opis uruchomienia}
W celu łatwego uruchomienia i konfiguracji projektu zastosowany został system automatyzacji budowy projektu \textit{Gradle}. Został on skonfigurowany w ten sposób, że aplikację można łatwo uruchomić poniższym poleceniem. Przykład zakłada, że lokalizacja terminala jest ustawiona na katalog \textbf{zawierający projekt}.

\begin{lstlisting}[language=bash]
./gradlew run 
\end{lstlisting}

Dzięki zastosowaniu \textit{Gradle wrapper} nie jest wymagana instalacja narzędzia \textit{Gradle}.

\section{Modernizacja kodu}
Oryginalny kod był jednym z pierwszy moich projektów w języku Java. W celu uproszczenia rozwoju projektu oraz późniejszego korzystania z niego została przeprowadzona modernizacja zastosowanych technologii.

\subsection*{Biblioteki}
Początkowo projekt nie korzystał z narzędzia do budowy kodu a biblioteki były manualnie dołączane do ścieżki kompilatora. Podejście to zmienione na zastosowanie narzędzia \textit{Gradle}, które umożliwia automatyzację dołączania bibliotek oraz budowy aplikacji. Użyte biblioteki zostały również aktualizowane do nowych wersji.

\subsection*{Java 11}
Poziom języka został podniesiony z wersji 8 do 11. Umożliwiło to zastosowanie nowych funkcjonalności takich jak słowo kluczowe \texttt{var} oraz zastosowanie mechanizmu modułów.
Mechanizm modułów został wykorzystany głównie przy dołączaniu biblioteki \textit{Java FX}.

\subsection*{Strukrtóra kodu}
Zmieniona została strukrtóra modułów projektu, tak by była zgodna z panującymi standardami w większości projektów. Dodatkowo zasoby takie jak pliki \texttt{.fxml} zostały przeniesione do katalogu \texttt{resources}, co ułatwiło ich ładowanie.

\section{Opis zrównoleglenia}
% TODO: Opis zrównoleglenia

\section{Analiza uzyskanych wyników}
% TODO: Analiza uzyskanych wyników


\end{document}
